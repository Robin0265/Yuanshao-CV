% --------------------------
\documentclass{article}
%  8.3in x 11.7in
\usepackage[a4paper, total={6.3in, 9.7in}]{geometry}
\usepackage[affil-it]{authblk}
\usepackage{graphicx}
% Hyperlink
\usepackage[usenames,dvipsnames]{xcolor}
\usepackage{indentfirst}
\usepackage[unicode, draft=false]{hyperref}
% \definecolor{linkcolour}{rgb}{0,0.2,0.6}

% \hypersetup{colorlinks,breaklinks,urlcolor=linkcolour,linkcolor=linkcolour}
\graphicspath{{./images/}}
\usepackage{float}

% REFERENCE EXTENSIONS
\newcommand*{\fullref}[1]{\hyperref[{#1}]{\autoref*{#1} \nameref*{#1}}} % One single link
\newcommand*{\subsecref}[1]{\hyperref[{#1}]{Section \ref*{#1}, \nameref*{#1}}}
\newcommand*{\subsubsecref}[1]{\hyperref[{#1}]{Section , \nameref*{#1}}}
\newcommand*{\parref}[1]{\hyperref[{#1}]{\nameref*{#1}}}

\title{Research Proposal}

\author{Yuanshao Yang}
\affil{Department of Mechanical Engineering, Univereity of Michigan}
\date{}

\begin{document}

\maketitle


\section{Introduction}
% Overview of the research area, why prostheses are important
\begin{itemize}
    \item General Area to be studied
    
    Prostheses / Exoskeletons design has become an emerging field in recent years, due to its performance in assisting locomotion for people with disabilities \cite{PersonalizeMain}. 
    
    \begin{itemize}
        \item Personalization of Prosthesis / Orthoses / Exoskeleton design \cite{PersonalizeMain}
    \end{itemize}
    
    \item Why this area is important
    
    Due to the fact that many people suffer from locomotor defects, thus significantly modify the biomechanics and muscle activity of joints (e.g. asymmetrical gait patterns) \cite{GaitAdjustment}, \textbf{personalization} of prostheses / exoskeletons design becomes an emphasis of the field. Generally, the ultimate goal of prostheses / exoskeleton design is to personalize the experience in the correction of locomotion defects \cite{GaitCorrection}, and reduce the energy cost for long-distance locomotion \cite{ReduceEnergyCost}.

    \begin{itemize}
        \item many people suffer from locomotor defects, thus significantly modify the biomechanics and muscle activity of joints (e.g. asymmetrical gait patterns) \cite{GaitAdjustment}
        \item correct locomotion (e.g. gaits) defect for disabilities \cite{GaitCorrection}
        \item reduce energy cost for long-distance locomotion \cite{ReduceEnergyCost}
    \end{itemize}
\end{itemize}



\section{Research Questions}

\begin{itemize}
    \item what is already known in the field (with several critical studies)?
    
    % Many studies have revealed the importance of personalized design in prostheses / exoskeletons. 
    Specifically, recent studies have chosen 2 major approaches in the personalized design of prostheses / exoskeletons: unpowered exoskeleton \cite{UnpoweredExo} and powered exoskeleton \cite{OSL2020}. 

    \begin{itemize}
        \item common design goals for prostheses exoskeletons
        \begin{itemize}
            \item light-weight \cite{ProsthesisOverview}
            \item high energy density and efficiency \cite{OSLSpring}
        \end{itemize}
        \item Therefore: 2 approaches to design prostheses / exoskeletons
        \begin{itemize}
            \item unpowered exo \cite{UnpoweredExo}
            \item powered exo \cite{OSL2020}
        \end{itemize}
    \end{itemize}

    \item Why these studies are not sufficient, thus requiring my further research?
    
    However, trade-offs exist in both unpowered and powered exoskeleton design. While unpowered exoskeleton relieves the added mass penalty in exchange of a less power-density provided by actuation, powered exoskeleton is able to provide a larger power density but restricted by the added weight from Mechatronics parts \cite{tetheredExoBenefits}. Such trade-offs make it difficult to generalize the design for prostheses / exoskeleton devices, which also adds difficulty to personalize the experience. 
    
    The ultimate goal of prostheses / exoskeleton design is to be generalized in design and personalized in user experience. Therefore, a balance should be made between the benefits and drawbacks of different prostheses / exoskeleton design, and generalize the design for personalized prostheses / exoskeleton devices.

    \begin{itemize}
        \item both design are trade-offs: unpowered exo has a smaller added mass penalty but harder to provide large power-density, while autonomous exo is easier to provide power but difficult to offer light-weight features. 
        \item The ultimate goal of prostheses / exoskeleton design is to be generalized in design and personalized in user experience. 
    \end{itemize}
    
    \item \textbf{Key Research Questions, and Relevant Rationale}
    
    Therefore, the key research objective is put forward as follows: find a balance between the benefits and drawbacks of different prostheses / exoskeleton design, and generalize the design for personalized prostheses / exoskeleton devices.

    \begin{itemize}
        % Research Questions
        \item how to weigh the benefits of drawbacks of different prostheses / exoskeleton design, and generalize the design flow for personalized prostheses / exoskeleton devices? 
    \end{itemize}
    

\end{itemize}

\section{Plans and Methods}

\begin{itemize}
    
    \item Biomechanics: multiple ways to evaluate metabolic cost, by \textbf{Sensor Fusion} (respirometry, electromyography (EMG))
    
    The solution of the personalization design is two-folded. Firstly, the biomechanics of the user should be well understood, and the metabolic cost should be evaluated more precisely. This can be done by sensor fusion, which combines respirometry and electromyography (EMG) to evaluate the metabolic cost of the user. Such evaluation of the metabolic cost can provide a better understanding of the mechanics of human locomotion, thus contributing to a more detailed classification of locomotion modes and walking speeds in different situations.

    Such analysis on the biomechanics of human locomotion leads to a mixed strategy for prostheses / exoskeleton controls, which combines data-driven and classical control methods. The development of such a mixed control strategy should provide a more adaptable and dynamic movement for the user, thus reducing the energy cost of locomotion.
    
    \begin{itemize}
        \item Lead to: mixed control strategy (data driven + classical control)
        \item benefit: better understanding of mechanics (mid-level control side), contributes to the high level control (e.g. locomotion modes, various walking speed), make it more adaptable to dynamic movement.    
    \end{itemize}

    Secondly, both the method and the flow of Mechatronics system design should be optimized for generalizations of personalized prostheses / exoskeleton devices. On the one hand, variable stiffness and impedance should be applied to provide adjustable parameters, thus reducing the metabolic cost of locomotion. On the other hand, the components of the Mechatronics system should be adjustable \cite{OSLSpring} \cite{VariableMecheSys}, while maintaining control bandwidth \cite{SEABandwidth}. Such design should aim to provide net energy gain for the user, thus personalizing user experience in the correction of locomotion defects and reducing the energy cost for long-distance locomotion.

    \item Mechatronics (wearable robotics) design: variable stiffness and impedance (interchangeable components \cite{ProsthesisOverview}, compact energy storage devices \cite{OSLSpring}, impedance control \cite{VariableMecheSys}). 
    \begin{itemize}
        \item Goal: provide net energy gain for the user, thus reducing the metabolic cost of locomotion
        \item Components should be adjustable (e.g. series spring), while maintaining control bandwidth \cite{SEABandwidth}
    \end{itemize}
\end{itemize}

\section{Significance}
\noindent Overall, the research on the personalization of prostheses / exoskeleton design is significant in various aspects. On the one hand, the research can provide a more detailed understanding of the biomechanics of human locomotion, thus contributing to a more detailed classification of locomotion modes and walking speeds in different situations. On the other hand, the research generates a generalized method for personalizing prostheses / exoskeleton devices, while maintaining the function of reducing energy cost \cite{ReduceEnergyCost}, and correcting gait patterns \cite{GaitCorrection}.

\begin{itemize}
    \item personalized in \textbf{assistance}
    \begin{itemize}
        \item reduce energy cost \cite{ReduceEnergyCost}
    \end{itemize}
    \item personalized in \textbf{rehabilitation}
    \begin{itemize}
        \item correct gait patterns \cite{GaitCorrection}
    \end{itemize}
\end{itemize}

\bibliographystyle{ieeetr}
\bibliography{RPref}

\end{document}