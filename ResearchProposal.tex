%---------
\documentclass[8pt]{article}
%  8.3in x 11.7in
\usepackage[a4paper, total={7.3in, 10.7in}]{geometry}
\usepackage[affil-it]{authblk}
\usepackage{graphicx}
% Hyperlink
\usepackage[usenames,dvipsnames]{xcolor}
\usepackage{indentfirst}
\usepackage[unicode, draft=false]{hyperref}
% \definecolor{linkcolour}{rgb}{0,0.2,0.6}

% \hypersetup{colorlinks,breaklinks,urlcolor=linkcolour,linkcolor=linkcolour}
\graphicspath{{./images/}}
\usepackage{float}

% REFERENCE EXTENSIONS
\newcommand*{\fullref}[1]{\hyperref[{#1}]{\autoref*{#1} \nameref*{#1}}} % One single link
\newcommand*{\subsecref}[1]{\hyperref[{#1}]{Section \ref*{#1}, \nameref*{#1}}}
\newcommand*{\subsubsecref}[1]{\hyperref[{#1}]{Section , \nameref*{#1}}}
\newcommand*{\parref}[1]{\hyperref[{#1}]{\nameref*{#1}}}

\title{Research Proposal}

\author{Yuanshao Yang}
\affil{Department of Mechanical Engineering, Univereity of Michigan}
\date{}

\begin{document}

\maketitle


\section{Introduction}
% Overview of the research area, why prostheses are important
\begin{itemize}
    \item General Area to be studied
    \begin{itemize}
        \item Personalization of Prosthesis / Orthoses / Exoskeleton design \cite{PersonalizeMain}
    \end{itemize}
    \item Why this area is important
    \begin{itemize}
        \item many people suffer from locomotor defects, thus significantly modify the biomechanics and muscle activity of joints (e.g. asymmetrical gait patterns) \cite{GaitAdjustment}
        \item correct locomotion (e.g. gaits) defect for disabilities \cite{GaitCorrection}
        \item reduce energy cost for long-distance locomotion \cite{ReduceEnergyCost}
    \end{itemize}
\end{itemize}



\section{Research Questions}

\begin{itemize}
    \item what is already known in the field (with several critical studies)?
    
    \begin{itemize}
        \item common design goals for prostheses exoskeletons
        \begin{itemize}
            \item light-weight \cite{ProsthesisOverview}
            \item high energy density and efficiency \cite{OSLSpring}
        \end{itemize}
        \item Therefore: 2 approaches to design prostheses / exoskeletons
        \begin{itemize}
            \item tethered \cite{tetheredExoEg}
            \item autonomous \cite{OSL2020}
        \end{itemize}
    \end{itemize}
    
    \item Why these studies are not sufficient, thus requiring my further research?
    
    \begin{itemize}
        \item both design are trade-offs: tethered exo has a smaller added mass penalty \cite{tetheredExoBenefits}, while autonomous exo is easier to generate light-weight features \cite{OSLSpring}. 
        \item The ultimate goal of prostheses / exoskeleton design is to be generalized in design and personalized in user experience. 
    \end{itemize}
    
    \item \textbf{Key Research Questions, and Relevant Rationale}
    
    \begin{itemize}
        % Research Questions
        \item how to weigh the benefits of drawbacks of different prostheses / exoskeleton design for personalization? 
    \end{itemize}
    

\end{itemize}

\section{Plans and Methods}

\begin{itemize}
    \item Biomechanics: multiple ways to evaluate metabolic cost, by \textbf{Sensor Fusion} (respirometry, electromyography (EMG))
    \begin{itemize}
        \item Lead to: mixed control strategy (data driven + classical control)
        \item benefit: better understanding of mechanics (mid-level control side), contributes to the high level control (e.g. locomotion modes, various walking speed), make it more adaptable to dynamic movement.    
    \end{itemize}
    \item Mechatronics (wearable robotics) design: variable stiffness and impedance (interchangeable components \cite{ProsthesisOverview}, compact energy storage devices \cite{OSLSpring}, impedance control \cite{VariableMecheSys}). 
    \begin{itemize}
        \item Goal: provide net energy gain for the user, thus reducing the metabolic cost of locomotion
        \item Components should be adjustable (e.g. series spring), while maintaining control bandwidth \cite{SEABandwidth}
    \end{itemize}
\end{itemize}

\section{Significance}

\begin{itemize}
    \item personalized in \textbf{assistance}
    \item personalized in \textbf{rehabilitation}
\end{itemize}

\bibliographystyle{ieeetr}
\bibliography{RPref}

\end{document}