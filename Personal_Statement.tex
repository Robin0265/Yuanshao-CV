%---------
\documentclass[12pt]{article}

\usepackage[a4paper, total={7in, 10in}]{geometry}

% Hyperlink
\usepackage[usenames,dvipsnames]{xcolor}
\usepackage{indentfirst}
\usepackage[unicode, draft=false]{hyperref}
\definecolor{linkcolour}{rgb}{0,0.2,0.6}
\hypersetup{colorlinks,breaklinks,urlcolor=linkcolour,linkcolor=linkcolour}

\usepackage{graphicx}
\graphicspath{ {./images/} }
\usepackage{float}

\title{Research Portfolio}
\author{Yuanshao Yang}
\date{}

% -- START OF DOC -- % 
\begin{document}

\maketitle

\section{SCU - Freshman and Sophomore}
At the orientation on the first day of college, students around me signed up for various clubs with passion under the guidance of seniors. However, after talking and looking around, I didn’t sign any of the application forms. This was because my intention was to join only one club that was attractive enough for me to devote my energy and time to, and I had not encountered one like that today. 

Meanwhile, my college life began. I was overly excited by the variety of class formats that were completely different from high school, especially the group projects and research opportunities that allowed me to meet many people from different backgrounds and with different perspectives on science. We enjoyed discussing class projects as well as the topic of combining math and natural science with computers to solve real-world problems. Eventually, this summer, I co-founded the M.S.D.M (Math, Science, Data, and Modelling) Club with another classmate, with the goal of inviting more people with a common interest to develop our passions together. After organizing several seminars, we started to plan for the next year. 

I found the club I had been searching for. But the community to which I belong is larger than this club, it includes all the people who are interested in applied science and willing to devote their dedication to it. And I will stay in this community and continue to seek out such people, together, to unearth the wonders of sciences.

The interdisciplinary emphasis of the College of Engineering at the University of Michigan is a major appeal to me.

In March 2022, I began working at the West China Hospital Biomedical Big Data Center on the project “The Application of AI in Diagnosing Heart Diseases”. My mentor Dr. Xiaobo Zhou and I have spent the last several months working tirelessly to fabricate an AI model that is suitable for diagnosing heart diseases. In my opinion, this is a worthwhile endeavor that possesses the potential to make the precaution and diagnosis of heart diseases more efficient. Furthermore, it provides me with a platform upon which to integrate mathematics and mechanical automation systems with computer science and electronic information engineering. After developing AI models via Scikit-Learn and Pandas in Python and using training data sets that consisted of clinical diagnostic data of cardiac patients, I learned how to perform interpretability analysis using variable sensitivity analysis, thus leading to a more precise and maintenance-friendly deep learning model of making heat disease diagnosis.

This experience fueled my already-existing interest in automatics, and I am aware that in order to obtain more innovative insights into this topic, I must go beyond the typical mechanical engineering curriculum and develop more multidisciplinary abilities.

\section {UMich - Junior and Senior}

As a result, I am attracted by CoE at UMich since it empowered interdisciplinary strategies to provide solutions for civilization development. For example, one project launched by the Precision Systems Design Laboratory (PSDL) particularly entices my interest because it establishes a multidisciplinary modeling and design process for optimally integrating actuators or generators with the entire system of electromagnetic energy conversion in order to maximize efficiencies and eliminate redundancies. And Dr. Shorya Awtar, the director of PSDL, is one of the pioneering researchers whose work I admire and whom I wish for a chance to work with. 


And I believe the interdisciplinary curriculum of the ME program will prepare me for such a chance. First, I could build my upper-division study to focus on robotics with courses like Mechatronics and Robotics. In addition, I plan to take a minor in Electrical Engineering to gain more insights into control and communication. And most importantly, with the research opportunities that UMich offers for undergraduates, I would like to start my independent research project in my beloved field of robotics and automatics. 

After receiving my bachelor's degree, I aim to continue my research interest in automatics as a lifetime endeavor. I hope the ME program at UMich will enable me to take a step forward toward my goal. 

From the 

% % -- END OF DOC -- % 
\end{document}