%---------
\documentclass[8pt]{article}
%  8.3in x 11.7in
\usepackage[a4paper, total={7.3in, 10.7in}]{geometry}
\usepackage[affil-it]{authblk}
\usepackage{graphicx}
% Hyperlink
\usepackage[usenames,dvipsnames]{xcolor}
\usepackage{indentfirst}
\usepackage[unicode, draft=false]{hyperref}
\definecolor{linkcolour}{rgb}{0,0.2,0.6}

\hypersetup{colorlinks,breaklinks,urlcolor=linkcolour,linkcolor=linkcolour}
\graphicspath{{./images/}}
\usepackage{float}

% REFERENCE EXTENSIONS
\newcommand*{\fullref}[1]{\hyperref[{#1}]{\autoref*{#1} \nameref*{#1}}} % One single link
\newcommand*{\subsecref}[1]{\hyperref[{#1}]{Section \ref*{#1}, \nameref*{#1}}}
\newcommand*{\subsubsecref}[1]{\hyperref[{#1}]{Section , \nameref*{#1}}}
\newcommand*{\parref}[1]{\hyperref[{#1}]{\nameref*{#1}}}

\title{Academic Statement of Purpose}

\author{Yuanshao Yang}
\affil{Department of Mechanical Engineering, Univereity of Michigan}
\date{}

\begin{document}

\maketitle


\section{Background \& Motivation}

% RESEARCH QUESTIONS I AM INTERESTED IN
% A BROADER PROBLEM YOU WOULD LIKE TO SOLVE

% Research in Wearable Robotics Systems, including prostheses and exoskeletons, has made significant strides in aiding individuals with disabilities. Nevertheless, many of these systems do not provide a cohesive solution for addressing locomotor deficits in disabled individuals, primarily due to the absence of bidirectional feedback between human movement and Mechatronics system.

Research in Wearable Robotics Systems, including prostheses and exoskeletons, has made significant progresses in aiding people with disabilities \cite{ProsthesisOverview}. Nevertheless, many of these systems do not provide a robust solution for addressing locomotor deficits, primarily due to the absence of bidirectional feedback between human movement and Mechatronics system.

\section{Research Questions}

% Based on the background stated above, my research questions, which also serve for the focus of my research, are as follows:

The presence of these issues raises the following questions, which also serves as my research focus: 

\begin{itemize}
            
    % \item {How the fundamental actuator modes are combined, in order to mimic human gaits?}
    
    % --- KEY RESEARCH QUESTIONS CORRESPONDING SPECIFIC PROJECTS
    % --- Robot Swimmer Project --- % 
    \item {How sensing, reasoning, and acting mechanics are applied and interacted in bionic robotics systems?}
    % --- Open-Source leg Generalization --- %
    \item {How robotics systems design can be better optimized to provide a generalized, user-friendly experience in robotics system developments?}
    % --- Series Spring Project --- %
    \item {How designs in compact mechanical systems are performed, to balance tradeoffs and spans the design space in common wearable robotics applications?}
    
\end{itemize}

% A SHORT PARAGRAPH THAT SHOWS MOTIVATION

 

% 

\section{Related Projects}

I seek to analyze the dynamics of common biological system and develop corresponding robotics systems that allows more powerful, efficient interactions between human locomotion and Mechatronics system. In essence, my research helps to improve such interactions on the following aspects: 

\begin{itemize}

    \item {I help optimize \hyperref[sec:RobotSwimmer]{the design of robotics systems and their control strategies} from mechanical designs and computational simulations.}

    \item {I help optimize the design of the \hyperref[sec:OSL-Library]{robotics software systems} to create a more generalized platform in evaluating further mechanical designs and control strategies of wearable robotics systems.}

    \item {I help improve series elastic actuator (SEA) designs by the implementation of \hyperref[sec:series-spring]{torsional springs assemblies}, that both spans the mechanical design space as energy storage gadgets and maintains high performance in low stiffness behaviors \cite{SpringDesign_OSL}.}
    
    
    

\end{itemize}

% A SHORT PARAGRAPH OF THE ASPECTS OF MY RESEARCH SOLUTIONS

    \subsection{Design \& Control of Bionic Robot Swimmer}  \label{sec:RobotSwimmer}
    
    To address the question "How sensing, reasoning, and acting mechanics are applied and interacted in bionic robotics systems", I began my attempt in combining biomechanics principles and Mechatronics approaches by the design of a bionic robot swimmer model.

    In general, I designed a robot's dynamics model which senses obstacles in the nearby environment, plans corresponding paths to avoid obstacles, and acts to perform target tracking. 

    Firstly, I developed a mechanical model which mimics the bacteria with dual flagella in \textbf{Computer Aided Design (CAD)}, that is compatible with common sensors, and verified the design through \textbf{Computational Fluid Dynamics (CFD)} analysis, that such design will empower flexibility in motion to allow various postures of obstacle avoidance. Moreover, I implemented the A-star\cite{AstarAlgo} path planning algorithm to plan the path of the robot swimmer, and established an \textbf{inverse kinematics model} to estimate the proper actuation for a specific trajectory. Finally, I designed the control strategies through \textbf{feedback control and feedforward control}, respectively, to simulate the robot swimmer's performance object tracking and obstacle avoidance. Also, I analyzed the behavior of the linearized system and defined the margins of the controller design to reach stability though \textbf{root locus analysis}\cite{RootLocus}.

    Such design aims to offer a feasible solution for future research in control strategies for robot swimmers, in order to achieve better performances in targeting transportation. 
    
    However, such bionic robot swimmer model is only specialized for specific tasks, for example, targeting transportation. Therefore, I began my attempt in generalizing the software library of Open-Source Leg project, to investigate the relationship between software designs and hardware compatibility in robotics systems.

    \subsection{Software Library Design \& Generalization of Open-Source Leg}   \label{sec:OSL-Library}

    To address the question "How robotics systems design can be better optimized to provide a generalized, user-friendly experience in robotics system developments", I began my attempt in redesigning the software library of Open-Source Leg project.
    
    The Open-Source Leg project is a standardized hardware and software platform for prosthesis designs, controls, and tests. Such platform is expected to be flexible towards hardware choices. However, the previous versions of its software library is biased to a specific hardware choice used in one lab, which fails to close the gap between individual researches and adds difficulty in comparing experiment results. 

    % -- MOTIVATION: A MORE GENERALIZED OPENSOURCELEG -- %
    In general, I designed a base library for the actuator and sensor modules, that acts as templates for new devices and simplifies hardware integration. I also redeveloped the instance of actuator modules, and offered example benchmarking results based on the generalized library. Such generalized software library design will allow more contributes to sustainable developments, and offer a standardized platform for characterization, benchmarking, and evaluation processes. 

    Nevertheless, the generalized software library design fails to provide flexibility in mechanical components, especially the design with tradeoffs, which is essential for robotics system design. Therefore, I began my attempt in designing a series-spring system that can be used in the Open-Source Leg project, to provide more choices in mechanical components.

    % -- MOTIVATION: A MORE GENERALIZED OPENSOURCELEG -- %
    
    \subsection{Series-Spring Design \& Evaluation of Open-Source Leg}  \label{sec:series-spring}

    
    % -- MOTIVATION: LINKING SPRINGS IN SERIES-- %
    To address the question "How designs in compact mechanical systems are performed, to balance tradeoffs and spans the design space in common wearable robotics applications", I began my attempt in designing a series-spring system that is primarily used in the Open-Source Leg hardware design, to explore the design space of mechanical components.

    In general, I designed a spring architecture that links multiple torsional springs in series through flat head screws, which simplifies the configuration of series springs. I designed the strategy to measure the deflection of the spring through computer vision. I also evaluated the backlash effect caused by the connections at each springs. Such design will allow a larger energy capacity and better low-stiffness behavior compared with single springs, thus spanning the design space of series elastic actuator components. 

    Still, the series-spring design only focuses on the spring designs, which has a limited impact on offering a more flexible design of all hardware choices in series elastic actuator structures. Further research is needed to span the design space further in common series elastic actuator systems, from software to hardware designs.
    

    

\section{Future Agenda} \label{sec:Agenda}

In general, my current research mainly focus on the construction of fundamental wearable robotics systems and the generalization of designs. Nevertheless, my research lacks the generalization in generalizing hardware designs and control strategies, which is essential in constructing robotics systems that is more generalized to user applications. 

Specifically, I am interested in continuing my research in the following aspects:

\begin{itemize}

    \item {How control strategies can be generalized to provide a robust solution for reference tracking and disturbance rejection}
    \begin{itemize}
        \item {Specifically, how control strategies can be optimized for addressing locomotor deficits in disabled individuals \cite{OSL2020}. }
    \end{itemize}
    \item {How series elastic actuator designs can be optimized to provide more flexibility for software and hardware designs}
    \begin{itemize}
        \item {Specifically, how the series elastic actuator designs can be generalized for energy storage and shock resistance in common wearable robotics systems.}
    \end{itemize}

\end{itemize}



\section{Why Grad School?} \label{sec:GradSchool}
To be continued ... 

\bibliographystyle{ieeetr}
\bibliography{StatementRef}

\end{document}